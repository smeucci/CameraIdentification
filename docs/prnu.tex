\section{Il \emph{pattern noise}}

Durante tutto il processo di acquisizione ed elaborazione, l'immagine può essere affetta da rumore. 
In particolare, ogni fotocamera digitale possiede un sensore
lievemente differente dalle altre dello stesso modello, per via di un \emph{disturbo}
univoco e riconoscibile, ciò permette di identificare il sensore a partire dalle
immagini, sfruttando lo stesso principio della balistica nel riconoscimento dei segni lasciati dai proiettili nelle armi.

Il disturbo generato nelle immagini è legato sia al sensore che ai dettagli costruttivi. Questo
assicura una differenza tra singoli dispositivi sufficiente a rendere improbabile
la presenza di due camere che generino il medesimo disturbo, proprio come avviene per le impronte digitali. 

Rumori di questo genere sono di tipo \emph{pattern noise}. Quest'ultimo è composto da due parti: il \emph{Fixed Pattern Noise} (FPN) ed il rumore \emph{Photo-Response Non-Uniformity} (PRNU).

FPN è un rumore casuale e può essere automaticamente rimosso sottraendo all'immagine originale un'immagine scura. Anche il PRNU può essere scomposto nel rumore \emph{Pixel Non-Uniformity} (PNU) e componenti a bassa frequenza, dove il PNU è il risultato della differenza di luminosità dei pixel a causa delle regioni non omogene del sensore e delle imperfezioni nella costruzione dello stesso.

Due immagini acquisite dallo stesso sensore CCD generalmente presentano lo stesso rumore PNU; inoltre il PNU non è soggetto a variazioni di temperatura o umidità il che lo rende adatto allo scopo dell'identificazione della sorgente.

Il rumore residuo di un'immagine $W$ può essere espresso come
$$
W = I - F(I) = IK - \Xi
$$
dove $F()$ è una funzione di \emph{denoising}, $K$ è una costante e $\Xi$ è un insieme di fattori di rumore.

A questo punto è necessario stimare $K$ a partire da un insieme $N$ di immagini ottenute dalla medesima camera, assumendo che $\Xi$ sia di tipo gaussiano. Per ciascun $k = 1, \ldots, N$:
$$
\frac{W_k}{I_k} = K + \frac{\Xi_k}{I_k}
$$
La \emph{log-likelihood} di $\frac{W_k}{I_k}$, dato $K$, è calcolata con
$$
L(K) = -\frac{N}{2} \sum_{k = 1}^{N} log(\frac{2\pi\sigma^2}{I_{k}^{2}}) -  \sum_{k = 1}^{N} \frac{(W_k / I_k - K)^2}{2\sigma^2 / I_{k}^{2}}
$$
Si risolve l'equazione $\frac{\delta L(K)}{\delta K} = 0$ per ottenere la stima $\hat{K}$ di $K$:
$$
\hat{K} = \frac{\sum_{k = 1}^{N} W_k I_k}{\sum_{k = 1}^{N} I_{k}^{2}}
$$



cosa è il prnu in generale (non nel dettaglio) e come si calcola la pce tra due prnu

dettagli della estrazione: resize immagini (alla dimensione inferiore), rotazione immagini in landscape mode